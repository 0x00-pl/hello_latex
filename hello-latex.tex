\documentclass[11pt,a4paper]{article}
\usepackage[final]{listings}
\usepackage{color}
\usepackage{graphicx}
\usepackage{CJK}
\usepackage{rotating}
\usepackage{amsmath}
\usepackage{amssymb}
\usepackage{amsthm}

\begin{document}

\title{this is a title}
\author{pl and 0x00-pl}  % optional
\date{the date is today}  % optional
\maketitle

\begin{CJK}{UTF8}{gbsn}
\tableofcontents
\end{CJK}

%\mainmatter % only for some template

\begin{CJK}{UTF8}{gbsn}
  \section{section 0 哈哈 \LaTeX}{
    \LaTeX
  }
\end{CJK}

\section{compile}{
  \subsection{MakeFile}{
    latex hello-latex.text\\
    dvipdfm hello-latex.dvi
  }
  \subsection{sections}
  \textbackslash{}section\\
  \textbackslash{}subsection\\
  \textbackslash{}subsubsection\\
  \textbackslash{}paragraph\\
  \textbackslash{}subparagraph

  \subsubsection{set section counter}
  \textbackslash{}setcounter\{section\}\{3\}
}

\begin{CJK}{UTF8}{gbsn}
  \section{contents 你好 \LaTeX}{
    \linespread{1.3} % line height
    \paragraph{first contents}
    你好 \LaTeX
    
    You can mix latin letters and chinese.
  }
\end{CJK}

\section{strings}{
  this is some text of content of ``this'' subsection.\\
  this is another line of content.
}

\section{formulas}{
  \centering{B$^2$ = {\bf B} $\times$ B}
  \begin{center}
    A[3] is $\mathrm{A}_3$
  \end{center}
}


\section{left and right}{
  \raggedright{\textbackslash{}raggedright\{text on left\}}
  \begin{flushleft}
    \textbackslash{}begin\{flushleft\}\\
    flushleft\\
    \textbackslash{}end\{flushleft\}
  \end{flushleft}
  \raggedleft{\textbackslash{}raggedleft\{text on right\}}
  \begin{flushright}
    \textbackslash{}begin\{flushright\}\\
    flushright\\
    \textbackslash{}end\{flushright\}
  \end{flushright}
  \centering{\textbackslash{}centering\{centering\}}
  \begin{center}
    \textbackslash{}begin\{center\}\\
    center\\
    \textbackslash{}end\{center\}
  \end{center}
}

\section{something verbatim}{
\begin{verbatim}
some thing verbatim, which without escape \\ /23454234$&%^&^(*^\z@%@#$%
\end{verbatim}
}

\section{colors}{
  \definecolor{tianyi_blue}{RGB}{102,204,255}
  \begin{figure}[!h]
    \centering
    \begin{turn}{45}
      \includegraphics[width=0.5\textwidth,bb=0 0 140 132]{head_image.jpg}
    \end{turn}
    \caption{\color{tianyi_blue} title of image}
  \end{figure}
}
\clearpage

\section{images}{
  \begin{figure}[h!]
    \begin{minipage}[b]{0.45\linewidth}
      \centering
      \includegraphics[natwidth=\linewidth,natheight=0.5\linewidth,scale=1]{head_image.jpg}
    \end{minipage}
    \begin{minipage}[b]{0.45\linewidth}
      \centering
      \includegraphics[natwidth=\linewidth,natheight=0.5\linewidth,scale=1]{head_image.jpg}
    \end{minipage}
    \label{fig:2img}
    \caption{2 images}
  \end{figure}
}

\section{size and fount}{
  \subsection{size}{
    \tiny tiny\\
    \footnotesize{footnotesize}\\ 
    {\small small}\\
    \normalsize normalsize\\
    \large large\\
    \Large Large\\
    \LARGE LARGE\\
    \huge huge
  }
  \subsection{fount}{
    \textnormal{testnormal} {\normalfont normalfont}\\
    \texttt{texttt} {\ttfamily ttfamily}\\
    \textbf{textbf} {\bfseries bfseries}\\
    \textit{testit} {\itshape itshape}
  }
}

\section{lists}{
  \begin{itemize}
  \item item 1
  \item item 2
  \end{itemize}
}

\section{turn and rotate}{
  see the difference:\\
  prev line\\
  turn
  \begin{turn}{30}
    --30--
  \end{turn}
  deg\\
  and rotate
  \begin{rotate}{30}
    --30--
  \end{rotate}
  deg\\
  nextline
}

\section{tables}{
  \subsection{tabular}
  this is a tabular: \\
  \begin{tabular}{||lcr|lcr}
    \hline
    \multicolumn{3}{c|}{3$\times$3 matrix} & ex\\
    \hline
    11 & 12 & 13 \\ % symbol & to align contents
    \hline
    21 & 22 & 23 &1111&1111&1111\\
    \cline{1-2}\cline{4-5}
    31 & 32 & 33 & 34 & 35 & 36
  \end{tabular}
}
\section{vector image: gnuplot}{
  \subsection{set format}{
    set term postscript eps enhanced color font 'Times. 24'
    set output 'sin.eps'
  }
  \subsection{plot sin}{
    plot sin(x)
  }
  \subsection{save file}{
    set term x11
    set output
  }
  \subsubsection{import}{
    use graphicx package and includegraphics command
  }
}


\section{notation and referance}{
  \subsection{structure}{
    {
\begin{verbatim}
  \lable{marker}  % define marker
  ...
  \ref{marker}  % ref marker
  ...
  \pageref{marker}  % page of marker
\end{verbatim}
    }
    we can ref the (figure:2img): \ref{fig:2img} \\
    at page: \pageref{fig:2img}
  }

  \subsection{more on figure}{
    on\begin{verbatim}\begin{figure}[!t]\end{verbatim}
    the $[!t]$ notation means force the image shows on the top of the page. \\
    ps: \\
    $[!h]$ means put it here. \\
    $[!t]$ means put it on top. \\
    $[!b]$ means put it on bottom. \\
    $[!p]$ means put it into another page. \\
  }

  \subsection{footnote}{
    this is the footnote \footnote{the footnote text is here}
  }
  \subsection{hyper link}{
    %\usepackage{hyperref}
    use the package 'hyperref' \\
    $\backslash url\{http://the.url.is.here\}$ \\
    $\backslash href\{http://there.is.another.url\}$
    $\{$the link title or some discription$\}$
  }
}
  
\section{common errors -- part 4 of the lecture}{  


  \subsection{errors and warings}{
    $\{\}$ miss match: Too many $\}$'s \\
    letter wrong like $\backslash dtae\{Mar.2014\}$: Undifined control sequence\\
    not in math mode: Not in Mathematics Mode. \\
    image or box are too large: Bad Boxes! \\
    missing packages: Missing Packages
  }
  
  \subsection{lengths}{
    normally don't override textlength setting or something like that.
  }
  \subsection{counters}{not normally used}
  \subsection{boxes}{
    add box on the text or \framebox something \\
    use $\backslash$framebox
  }
  \subsection{rules and struts}{
    make some area black or something like that \\
    using \textbackslash{}rule command
  }
}


\section{mathematics and computers -- part 5 of the lecture}{
  \subsection{methmatics packages and useage}{
    \textbackslash{}usepackage\{amsmath\}\\
    \textbackslash{}usepackage\{amssymb\}\\
    \textbackslash{}usepackage\{amsthm\}\\
    inline math:\\
    \(a+a\) \& $b+b$\\
    multi line:\\
    \[c+c\] \& $$d+d$$\\
    \begin{proof}
      proof of something.\\
      proof detail.
    \end{proof}
  }
  \subsection{computers heigh light}{
    \textbackslash{}uspackage\{listings\}\\
    \textbackslash{}uspackage\{color\}\\
    
    \textbackslash{}definecolor\{myred\}\{rgb\}\{0.5, 0, 0\}\\
    \textbackslash{}definecolor\{mygreen\}\{rgb\}\{0, 0.5, 0\}\\
    \textbackslash{}definecolor\{myblue\}\{rgb\}\{0, 0, 0.5\}\\

    \definecolor{myred}{rgb}{0.5, 0, 0}
    \definecolor{mygreen}{rgb}{0, 0.5, 0}
    \definecolor{myblue}{rgb}{0, 0, 0.5}
    \lstdefinestyle{customc}{
      belowcaptionskip=1\baselineskip,
      breaklines=true, frame=L, xleftmargin=\parindent,
      language=Java, showstringspaces=false,
      basicstyle=\footnotesize\ttfamily,
      keywordstyle=\bfseries\color{mygreen},
      identifierstyle=\color{myblue},
      stringstyle=\color{myred}
    }
    \lstset{style=customc}
    
    \begin{lstlisting}[frame=single]
      public class Factorial
      {
        public static void main(String[] args)
        { ... }
        public static int factorial(int n)
        { ... }
      }
    \end{lstlisting}
  }
  \subsection{}{}
  \subsection{}{}
}


\end{document}
